\documentclass[aps,pra,12pt,notitlepage,tightenlines]{revtex4-1}
\usepackage[margin=2cm]{geometry}
\usepackage{amsmath,amssymb,textcomp,graphicx,url,bm,lipsum,hyperref,color,subcaption,afterpage,gensymb}

\begin{document}

\title{Research Plan\vspace{0mm}}
\author{Wilf Shorrock\vspace{1mm}}
\affiliation{Imperial College London\vspace{-0.5mm}}
\date{June 30, 2018}

\maketitle

Following on from my work on the ND280 software, where I helped look at the different options for the software's version control, its package naming, and the implementation of ROOT6, I plan to help with the transition of the software from the CVS (Concurrent Versions System, a centralised version control system) repositories to GitLab (a distributed version control system). This will include implementing validation tests that check any uploaded code to ensure the software still runs correctly. I'm in a good position to carry out this work as I have access to the COMET repositories on GitLab and I'm in contact with several COMET collaborators (COMET is another particle physics experiment with involvement from Imperial) that can give advice on how they transitioned to GitLab. 

Once the code is transitioned it will also need to be maintained. It is still to be decided whether a CVS build will be kept frozen or will have anything added to the GitLab repositories copied over to it, in which case it could be my duty to transfer data added to GitLab over to CVS. Any other issues that arise, such as updating package names, could also be assigned to me.

Currently, the ND280 software is built using CMT (Configuration Management Tool), which creates the many packages that comprise the software. It has been decided that we will move on to using CMake instead of CMT, due to its flexibility and efficiency. It is still to be determined whether this transition will happen during the migration to GitLab or after. In either case, I will be able to help implement the CMake build. I have some experience of the CMake language after studying the COMET implementation of their CMake build.

I am becoming familiar with the new Super-FGD detector to be installed for the ND280 upgrade in about 2 years. I assisted with the beam test of the prototype at CERN, which involved assembling the electronics of the detector and taking data during the test. This could lead on to me helping with the analysis of the data from the beam test. From there, I may also become involved with the actual Super-FGD and help to get it ready for installation at ND280.

I will also progress with my work on the Production 7 (a Production refers to the version of the NEUT simulation program used to create simulation data for the study) analysis of electron neutrinos and anti-electron neutrinos at ND280 for the anti-neutrino beam data. I have read up on the Production 6 analysis and am now in the process of reproducing their selection plots. My future work will move on to using the new Production 7 Monte Carlo data in my analysis and extracting the electron neutrino and anti-electron neutrino composition of the beam and the detector systematic errors by performing fits on the data after selection cuts. I will also validate the results of the fits using Monte Carlo data.

Another potential project for me is the chance to perform an oscillation analysis using the latest data from T2K. This could form quite a large portion of the work included in my thesis, so it is important I become more familiar with the techniques and knowledge required for such an analysis.

After joining the T2K group at Imperial College London, I have learnt much about the experiment and gained experience working with software and hardware. This plan has outlined how I will progress with my PhD by building upon my initial work and taking advantage of potential research opportunities in the future. It may be the case that the plan will not turn out to be entirely accurate, as other projects may unexpectedly present themselves or some of the proposed projects may fall through or be delayed, but there are plenty of exciting options to keep me busy!

\end{document}