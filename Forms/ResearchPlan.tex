\documentclass[aps,pra,12pt,notitlepage,tightenlines]{revtex4-1}
\usepackage[margin=2cm]{geometry}
\usepackage{amsmath,amssymb,textcomp,graphicx,url,bm,lipsum,hyperref,color,subcaption,afterpage,gensymb}

\begin{document}

\title{Research Plan\vspace{0mm}}
\author{Wilf Shorrock\vspace{1mm}}
\affiliation{Imperial College London\vspace{-0.5mm}}
\date{June 30, 2018}

\maketitle

Following on from my work on the ND280 software, I plan to help with the transition of the software from the CVS (Concurrent Versions System) repositories to GitLab, be it CMT (Configuration Management Tool) or CMake. This will include implementing the build method for the code on GitLab. This should be completed by the end of 2018.

Once the code is transitioned it will also need to be maintained. It is still to be decided whether a CVS build will be kept frozen or will have anything added to the GitLab repositories copied over to it, in which case it could be my duty to transfer data added to GitLab over to CVS. Any other issues that arise, such as updating package names, could also be assigned to me. This maintenance will most likely last until the end of my PhD, although the work load should reduce once the initial transition has been completed.

I will also progress with my work on the updated analysis of electron neutrinos and anti-electron neutrinos at ND280 for the anti-neutrino beam data. I have read up on the previous analysis and am now in the process of reproducing their selection plots. My future work will move on to using the updated Monte Carlo data and performing an analysis for $\bar{nu}_e$ CC 0$\pi^0$ interactions resulting in an updated cross-section measurement. My ultimate goal for this study is to be included as an author on the resulting paper, which should take about 8 months of work, say August 2018 - March 2019.

I am becoming familiar with the new Super-FGD detector to be installed for the ND280 upgrade in about two years. I assisted with the beam test of the prototype at CERN, which involved assembling the electronics of the detector and taking data during the test. This could lead on to me helping with the analysis of the data from the beam test. From there, I may also become involved with the actual Super-FGD and help to get it ready for installation at ND280. The beam test analysis will most likely last until the end of the year, with more data being taken during August. Hence, analysis will probably last from now up until mid-2019.

Looking at more long-term options that will make up the bulk of my thesis, I plan to undertake an analysis/task that combines my recent work on the ND280 upgrade and the electron neutrino contamination analysis at the near detector. One possible path, and one I have set as my main project for the future, is to help with software preparation for the new ND280 detectors, such as implementing the calibration methods, before they are used in ND280 itself. This will make use of the results from the beam tests of the upgraded detectors, which are being carried out this year. For this software preparation, I will be working with Dr. Per Jonsson, who is Calibration Convener at ND280. This could start after the analysis of the beam test data, so should last from mid-2019 to until the detectors are installed in 2020. 

There is also the possibility of preparing an analysis for the new detectors. My PhD will end before the detectors are installed and an analysis of their data can be performed, but a Future Analysis Task Force, led by Dr. Phillip Litchfield, will provide recommendations for the T2K Collaboration on how future oscillation analyses could proceed. With help from colleagues in one of the main analysis groups---Mach3---I could help implement these analysis preparations, with either the existing detectors or the upgraded detectors. Work on this could also start mid-2019 and last up to the end of my PhD in 2021.

\end{document}